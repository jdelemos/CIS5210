
\documentclass{article}
\usepackage[utf8]{inputenc}
\usepackage{courier}

\title{Penn CIS 5210 - Chapter 2}
\author{Jonathon Delemos - Dr. Chris Callison Burch}
\date{\today}

\begin{document}

\maketitle

This course investigates algorithms to implement resource-limited knowledge-based agents which
sense and act in the world. Topics include, search, machine learning, probabilistic reasoning, natural
language processing, knowledge representation and logic. After a brief introduction to the language,
programming assignments will be in Python.
— Description of CIS 4210/5210 in course catalog

\section{Chapter Two: Intelligent Agents}
\subsection{2.1 - Agents and Environments}
\begin{quote}
    \texttt{\textbf{Agent} is anything that can be viewed as perceiving its environment through sensors and acting upon the environment through \textbf{actuators}.
        \\ Agent behavior is mapped through a Mathematical function called \textbf{agent functions.} Internally, we the agent function is looks at as a \textbf{agent program.}
        \\The example in the book they use is vaccuum world.}
\end{quote}
\end{document}
