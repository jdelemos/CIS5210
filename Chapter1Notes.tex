 
\documentclass{article}
\usepackage[utf8]{inputenc}
\usepackage{courier}

\title{Penn CIS 5210 - Chapter 1}
\author{Jonathon Delemos - Dr. Chris Callison Burch}
\date{\today}

\begin{document}

\maketitle

This course investigates algorithms to implement resource-limited knowledge-based agents which
sense and act in the world. Topics include, search, machine learning, probabilistic reasoning, natural
language processing, knowledge representation and logic. After a brief introduction to the language,
programming assignments will be in Python.
— Description of CIS 4210/5210 in course catalog

\section{Section One: Introduction}
\subsection{What is AI?}
\begin{quote}
    \texttt{The field of artificial intelligence (AI) is concerned with not just understanding intelligence, but also building intelligent entities—machines that compute how to act effectively and safely in a variety of novel situations.
        We can define intelligence as the better part of rationality-loosely speaking, the ability to do the right thing at the right time.}
\end{quote}
\subsection{Acting Humanly: The Turing Test}
\begin{quote}
    \texttt{\textbf{Natural Language Processing} is the computers ability to communicate successfully in a humnan language.\\ \textbf{Knowledge Representation}
        is the ability to store what it knows or hears. \\ \textbf{Automated Reasoning} to answer questions and to draw new conclusions. \\ \textbf{Machine Learning} to adapt to new circumstances and to detect and extrapolate patterns. \\}
    \texttt{\textbf{Computer Vision} to perceive objects and scenes. \\ \textbf{Robotics} to move and manipulate objects. \\}
\end{quote}
\subsection{Thinking Humanly: Cognitive Modeling}
\begin{quote}
    \texttt{\textbf{Introspection} is the process of examining one's own thoughts and feelings.}
    \texttt{\textbf{Psychological Experiments} is observing a person in action. \\ \textbf{Brain Imaging} is observing the brain in action
        \\ \textbf{Cognitive Science} brings together the field Psychology and computer models to construct precise and testable theories of the human mind.
    }
\end{quote}
\subsection{Thinking Rationally: The Laws of Thought Approach}
\begin{quote}
    \texttt{\textbf{Logic} is a pattern for argument that is irrefutable. \\ If we don't have something that is absolutely certain, what we can do is fill that gap with probability analysis.}
\end{quote}
\subsection{Acting Rationally: The Rational Agent Approach}
\begin{quote}
    \texttt{\textbf{Agent} is something that acts. Of course all computer programs do something, but agents are expected to operate autonmously, perceive their environment, etc.
        \\ \textbf{Rational Agent} is a model that acts to achieve the best outcome, or best expected outcome. Methods are based on probability-AI has focused on the study and construction of agents that do the right thing. \\
        Using rational agents to accomplish these best possible outcomes is referred to as the \textbf{Standard Model}. \\
        \textbf{Limited Rationality} is the act of acting appropriately when there isn't enough time to calculate all the possible computations.}
\end{quote}
\subsection{Beneficial Machines}
\begin{quote}
    \texttt{The goal of this section is to define the multiple parameters that need to be weighed when making an agent. For example, in a self-driving car, one might
        the objective is simply to make it to your destination. However, you might be concerned with injury to passengers, annoying other drivers, following the rules of the road, etc.
        \\ This is where \textbf{Value Alignment Problems} come in. The values of the human must be aligned with the values of the agent. The result of not properly
        placing limits on value assignments is a computer doing seemingly irrational things to reach it's sole goal.}
\end{quote}
\subsection{The Foundations of Artifical Intelligence}
\begin{quote}
    \texttt{\textbf{Dualism} is the counter argument against that if the mind is governed by physical laws, we have no more free will than a rock falling down. This claims our mind is exempt from physical laws.
        There are loads of other philosophies here. Most of which are very interesting. Refer to pages 24-25. Moving forward. \\
        \textbf{General Problem Solver} is another term for greedy algorithm. \textbf{Example} Hello, World! This is an example for Sai. }
\end{quote}
\subsection{Mathematics}
\begin{quote}
    \texttt{\textbf{Tractability} is the word to describe problems at approach incomputability. This section deals with basic definitions of NP-complete, probability, and incompleteness.
        \\ \textbf{Markov Decision Processses} is the term used to describe sequential decision problems. This is when the first decision doesn't produce the best result.}
\end{quote}

\end{document}
